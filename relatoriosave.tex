% Options for packages loaded elsewhere
\PassOptionsToPackage{unicode}{hyperref}
\PassOptionsToPackage{hyphens}{url}
\documentclass[
]{article}
\usepackage{xcolor}
\usepackage[margin=1in]{geometry}
\usepackage{amsmath,amssymb}
\setcounter{secnumdepth}{-\maxdimen} % remove section numbering
\usepackage{iftex}
\ifPDFTeX
  \usepackage[T1]{fontenc}
  \usepackage[utf8]{inputenc}
  \usepackage{textcomp} % provide euro and other symbols
\else % if luatex or xetex
  \usepackage{unicode-math} % this also loads fontspec
  \defaultfontfeatures{Scale=MatchLowercase}
  \defaultfontfeatures[\rmfamily]{Ligatures=TeX,Scale=1}
\fi
\usepackage{lmodern}
\ifPDFTeX\else
  % xetex/luatex font selection
\fi
% Use upquote if available, for straight quotes in verbatim environments
\IfFileExists{upquote.sty}{\usepackage{upquote}}{}
\IfFileExists{microtype.sty}{% use microtype if available
  \usepackage[]{microtype}
  \UseMicrotypeSet[protrusion]{basicmath} % disable protrusion for tt fonts
}{}
\makeatletter
\@ifundefined{KOMAClassName}{% if non-KOMA class
  \IfFileExists{parskip.sty}{%
    \usepackage{parskip}
  }{% else
    \setlength{\parindent}{0pt}
    \setlength{\parskip}{6pt plus 2pt minus 1pt}}
}{% if KOMA class
  \KOMAoptions{parskip=half}}
\makeatother
\usepackage{graphicx}
\makeatletter
\newsavebox\pandoc@box
\newcommand*\pandocbounded[1]{% scales image to fit in text height/width
  \sbox\pandoc@box{#1}%
  \Gscale@div\@tempa{\textheight}{\dimexpr\ht\pandoc@box+\dp\pandoc@box\relax}%
  \Gscale@div\@tempb{\linewidth}{\wd\pandoc@box}%
  \ifdim\@tempb\p@<\@tempa\p@\let\@tempa\@tempb\fi% select the smaller of both
  \ifdim\@tempa\p@<\p@\scalebox{\@tempa}{\usebox\pandoc@box}%
  \else\usebox{\pandoc@box}%
  \fi%
}
% Set default figure placement to htbp
\def\fps@figure{htbp}
\makeatother
\setlength{\emergencystretch}{3em} % prevent overfull lines
\providecommand{\tightlist}{%
  \setlength{\itemsep}{0pt}\setlength{\parskip}{0pt}}
\usepackage[T1]{fontenc}
\usepackage[utf8]{inputenc}
\usepackage[portuguese]{babel}
\usepackage{booktabs}
\usepackage{float}
\usepackage{graphicx}
\usepackage{bookmark}
\IfFileExists{xurl.sty}{\usepackage{xurl}}{} % add URL line breaks if available
\urlstyle{same}
\hypersetup{
  pdftitle={Relatório - Ciências da Natureza},
  hidelinks,
  pdfcreator={LaTeX via pandoc}}

\title{Relatório - Ciências da Natureza}
\author{}
\date{\vspace{-2.5em}2025-11-11}

\begin{document}
\maketitle

\{r setup, include=FALSE\} knitr::opts\_chunk\$set( echo = FALSE,
message = FALSE, warning = FALSE, fig.pos = ``H'', fig.path =
``relatorio-figs/'', fig.crop = FALSE, \# evita pdfcrop/ghostscript dev
= ``pdf'' \# usa o dispositivo PDF base (independe de X11/Cairo) )

\section{Limpeza opcional da pasta de figuras antes de
renderizar}\label{limpeza-opcional-da-pasta-de-figuras-antes-de-renderizar}

try( unlink( list.files(``relatorio-figs'', pattern =
``\^{}(reg\textbar grafico\textbar hist).*\textbackslash.(pdf\textbar png)\$``,
full.names = TRUE), force = TRUE ), silent = TRUE )

\{r setup-e-carga, message=FALSE, warning=FALSE\} library(ggplot2)
library(kableExtra) library(dplyr)

amostra\_10p \textless- read.csv(``amostra\_10p.csv'', fileEncoding =
``latin1'')

\section{Presença na prova de Ciências da
Natureza}\label{presenuxe7a-na-prova-de-ciuxeancias-da-natureza}

amostra\_10p\(TP_PRESENCA_CN <- factor(
  amostra_10p\)TP\_PRESENCA\_CN, levels = c(0, 1, 2), labels =
c(``Ausente'', ``Presente'', ``Eliminado'') )

\noindent \textbackslash begin\{minipage\}{[}t{]}\{0.48\textwidth\}
\centering \textbf{Tabela 1:} Frequência - Presença na Prova de Ciências
da Natureza

\par\vspace{1ex}

\{\footnotesize \{r tabela-presenca-cn, results=`asis'\}
freq\_abs\_presenca\_cn \textless- table(amostra\_10p\$TP\_PRESENCA\_CN)
freq\_rel\_presenca\_cn \textless- prop.table(freq\_abs\_presenca\_cn)

tabela\_cn \textless- data.frame( \texttt{Presença\ na\ prova} =
names(freq\_abs\_presenca\_cn), Percentual = paste0(round(100 *
as.vector(freq\_rel\_presenca\_cn), 2), ``\%''), Frequência =
as.vector(freq\_abs\_presenca\_cn), check.names = FALSE )

tab\_cn \textless- knitr::kable( tabela\_cn, format = ``latex'',
booktabs = TRUE, longtable = FALSE )

cat(tab\_cn) \# solta o LaTeX direto

\} \textbackslash{} A partir da Tabela 1, observa-se a distribuição dos
participantes em relação à presença na prova de Ciências da Natureza,
destacando-se a proporção de estudantes que compareceram, se ausentaram
ou foram eliminados. Essa informação é importante para entender o
engajamento dos inscritos na aplicação da prova.
\textbackslash end\{minipage\} \hfill

\begin{minipage}[t]{0.48\textwidth}
\centering
\textbf{Gráfico 1:} Presença na Prova de Ciências da Natureza
\par\vspace{1ex}

{r grafico-presenca-cn, out.width="\\linewidth"}
df_presenca_cn <- as.data.frame(table(amostra_10p$TP_PRESENCA_CN))
colnames(df_presenca_cn) <- c("Situacao", "Frequencia")

# Calcular percentuais
df_presenca_cn$Percentual <- df_presenca_cn$Frequencia / sum(df_presenca_cn$Frequencia) * 100

ggplot(df_presenca_cn, aes(x = Situacao, y = Percentual, fill = Situacao)) +
  geom_bar(stat = "identity", show.legend = FALSE) +
  geom_text(aes(label = paste0(round(Percentual, 2), "%")), vjust = -0.5) +
  labs(
    x = "Situação",
    y = "Percentual (%)"
  ) +
  scale_fill_manual(values = c(
    "Ausente"   = "tomato",
    "Presente"  = "lightgreen",
    "Eliminado" = "skyblue"
  )) +
  theme_minimal() +
  theme(panel.grid.major.x = element_blank())

O Gráfico 1 apresenta visualmente a comparação entre estudantes presentes, ausentes e eliminados na prova de Ciências da Natureza, facilitando a interpretação da participação na avaliação.
\end{minipage}

\vfill

\noindent \textbackslash begin\{minipage\}{[}t{]}\{0.48\textwidth\}
\centering \textbf{Tabela 2:} Frequência Intervalar da Nota de Ciências
da Natureza

\par\vspace{1ex}

\{\footnotesize \{r tabela-nota-cn, results=`asis'\} \# Intervalos de 0
a 1000, de 100 em 100 breaks\_nota\_cn \textless- seq(0, 1000, by = 100)
labels\_nota\_cn \textless- paste(head(breaks\_nota\_cn, -1), ``--'',
tail(breaks\_nota\_cn, -1))

faixa\_nota\_cn \textless- cut( amostra\_10p\$NU\_NOTA\_CN, breaks =
breaks\_nota\_cn, include.lowest = TRUE, right = FALSE, labels =
labels\_nota\_cn )

freq\_abs\_cn \textless- table(faixa\_nota\_cn) freq\_rel\_cn \textless-
prop.table(freq\_abs\_cn)

tabela\_nota\_cn \textless- data.frame( \texttt{Intervalo\ de\ nota} =
names(freq\_abs\_cn), Percentual = paste0(round(100 *
as.vector(freq\_rel\_cn), 2), ``\%''), check.names = FALSE )

tab2\_cn \textless- knitr::kable( tabela\_nota\_cn, format = ``latex'',
booktabs = TRUE, longtable = FALSE )

cat(tab2\_cn)

\} \textbackslash{} A Tabela 2 apresenta a distribuição das notas de
Ciências da Natureza em classes de 100 pontos, permitindo identificar em
quais faixas se concentram os maiores e os menores desempenhos dos
estudantes. \textbackslash end\{minipage\} \hfill

\begin{minipage}[t]{0.48\textwidth}
\centering
\textbf{Gráfico 2:} Histograma das Notas de Ciências da Natureza com Curva Normal
\par\vspace{1ex}
{r hist-nota-cn-normal, out.width="\\linewidth", fig.asp=0.8}

# Média e desvio padrão das notas de CN
media_cn <- mean(amostra_10p$NU_NOTA_CN, na.rm = TRUE)
sd_cn    <- sd(amostra_10p$NU_NOTA_CN, na.rm = TRUE)

ggplot(amostra_10p, aes(x = NU_NOTA_CN)) +
  
  # Histograma em frequência relativa (%)
  geom_histogram(
    aes(y = after_stat(density * 100)),  # densidade convertida para %
    binwidth = 100,
    boundary = 0,
    fill = "steelblue",
    color = "white",
    alpha = 0.7
  ) +

  # Curva normal ajustada, também em %
  stat_function(
    fun = function(x) dnorm(x, mean = media_cn, sd = sd_cn) * 100,
    color = "red",
    linewidth = 1.2
  ) +

  labs(
    x = "Nota de Ciências da Natureza",
    y = "Frequência relativa (%)"
  ) +
  theme_minimal() +
  theme(
    panel.grid.major.x = element_blank(),
    plot.title = element_text(hjust = 0.5, face = "bold")
  )

\\
O Gráfico 2 compara a distribuição relativa das notas de Ciências da Natureza com uma curva normal ajustada pela média e pelo desvio padrão da amostra, permitindo avaliar se o desempenho se aproxima de um padrão aproximadamente normal ou se há assimetrias importantes.
\end{minipage}

\end{document}
