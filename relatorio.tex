% Options for packages loaded elsewhere
\PassOptionsToPackage{unicode}{hyperref}
\PassOptionsToPackage{hyphens}{url}
\documentclass[
  twocolumn]{article}
\usepackage{xcolor}
\usepackage[margin=1in]{geometry}
\usepackage{amsmath,amssymb}
\setcounter{secnumdepth}{-\maxdimen} % remove section numbering
\usepackage{iftex}
\ifPDFTeX
  \usepackage[T1]{fontenc}
  \usepackage[utf8]{inputenc}
  \usepackage{textcomp} % provide euro and other symbols
\else % if luatex or xetex
  \usepackage{unicode-math} % this also loads fontspec
  \defaultfontfeatures{Scale=MatchLowercase}
  \defaultfontfeatures[\rmfamily]{Ligatures=TeX,Scale=1}
\fi
\usepackage{lmodern}
\ifPDFTeX\else
  % xetex/luatex font selection
\fi
% Use upquote if available, for straight quotes in verbatim environments
\IfFileExists{upquote.sty}{\usepackage{upquote}}{}
\IfFileExists{microtype.sty}{% use microtype if available
  \usepackage[]{microtype}
  \UseMicrotypeSet[protrusion]{basicmath} % disable protrusion for tt fonts
}{}
\makeatletter
\@ifundefined{KOMAClassName}{% if non-KOMA class
  \IfFileExists{parskip.sty}{%
    \usepackage{parskip}
  }{% else
    \setlength{\parindent}{0pt}
    \setlength{\parskip}{6pt plus 2pt minus 1pt}}
}{% if KOMA class
  \KOMAoptions{parskip=half}}
\makeatother
\usepackage{graphicx}
\makeatletter
\newsavebox\pandoc@box
\newcommand*\pandocbounded[1]{% scales image to fit in text height/width
  \sbox\pandoc@box{#1}%
  \Gscale@div\@tempa{\textheight}{\dimexpr\ht\pandoc@box+\dp\pandoc@box\relax}%
  \Gscale@div\@tempb{\linewidth}{\wd\pandoc@box}%
  \ifdim\@tempb\p@<\@tempa\p@\let\@tempa\@tempb\fi% select the smaller of both
  \ifdim\@tempa\p@<\p@\scalebox{\@tempa}{\usebox\pandoc@box}%
  \else\usebox{\pandoc@box}%
  \fi%
}
% Set default figure placement to htbp
\def\fps@figure{htbp}
\makeatother
\setlength{\emergencystretch}{3em} % prevent overfull lines
\providecommand{\tightlist}{%
  \setlength{\itemsep}{0pt}\setlength{\parskip}{0pt}}
\usepackage{longtable}
\usepackage{etoolbox}
\usepackage[T1]{fontenc} \usepackage[utf8]{inputenc} \usepackage[portuguese]{babel} \usepackage{booktabs} \usepackage{float} \usepackage{graphicx}
\usepackage{booktabs}
\usepackage{longtable}
\usepackage{array}
\usepackage{multirow}
\usepackage{wrapfig}
\usepackage{float}
\usepackage{colortbl}
\usepackage{pdflscape}
\usepackage{tabu}
\usepackage{threeparttable}
\usepackage{threeparttablex}
\usepackage[normalem]{ulem}
\usepackage{makecell}
\usepackage{xcolor}
\usepackage{bookmark}
\IfFileExists{xurl.sty}{\usepackage{xurl}}{} % add URL line breaks if available
\urlstyle{same}
\hypersetup{
  pdftitle={Relatorio},
  pdfauthor={Fabio Firanzi, Heitor Dias, Julia Fideles, Matheus Soares, Tiago Braga},
  hidelinks,
  pdfcreator={LaTeX via pandoc}}

\title{Relatorio}
\author{Fabio Firanzi, Heitor Dias, Julia Fideles, Matheus Soares, Tiago
Braga}
\date{2025-11-11}

\begin{document}
\maketitle

\begin{tabular}{llr}
\toprule
Presença na prova & Percentual & Frequência\\
\midrule
Ausente & 26.74\% & 115880\\
Presente & 73.12\% & 316843\\
Eliminado & 0.13\% & 571\\
\bottomrule
\end{tabular}

\includegraphics[width=\linewidth]{relatorio_files/figure-latex/grafico-presenca-1}

\section{Tabela e gráfico de
presença}\label{tabela-e-gruxe1fico-de-presenuxe7a}

A tabela ``Frequência de presença na prova de LC'' e o Gráfico
``Presença na prova de LC'' exibem a porcentagem de alunos presentes,
ausentes e eliminados na prova de linguágens e códigos. Com base nessa
tabela, pode-se identificar que a grande maioria, 73,1\%, dos alunos
estava presente na avaliação, 26,7\% ausente e uma minoria de 0,1\% foi
eliminada antes, durante ou após a prova. Embora a maioria tenha
completado a avaliação, o alto índice de ausência sugere a necessidade
de estudos futuros para investigar os fatores que contribuem para essa
abstenção.

\includegraphics[width=\linewidth]{relatorio_files/figure-latex/hist-linguagens-1}

\section{Histograma da nota de LC}\label{histograma-da-nota-de-lc}

Para a criação desse histograma foi utilizado a nota dos alunos
presentes e não eliminados na avaliação de Linguagens e códigos, com
esse dado sendo analizado em sua frequência, média, desvio padrão e
distribuição.

O ``Histograma da Nota de LC'' ilustra a distribuição de frequência das
notas de Linguagens e Códigos para todos os alunos presentes. Ele foi
construído sobrepondo uma curva normal teórica (linha vermelha),
calculada a partir da média e do desvio padrão da amostra, sobre o
histograma das notas reais (barras azuis).

A maior concentração de notas é de 500-600 pontos, contendo
aproximadamente 55\% dos alunos, o desvio padrão das notas é de 69,87.

\begin{table}
\centering
\caption{\label{tab:unnamed-chunk-2}Tabela Comparativa: LC vs MT (para 298.976 alunos presentes em ambas e com nota > 0)}
\centering
\begin{tabular}[t]{lrrrrr}
\toprule
Prova & Média & Mediana & Desv. Padrão & Min & Máx\\
\midrule
LC & 526.70 & 533.3 & 68.03 & 298.8 & 795.8\\
MT & 527.24 & 499.1 & 113.91 & 342.8 & 961.9\\
\bottomrule
\end{tabular}
\end{table}

\section{Tabela e gráfico LC vs MT}\label{tabela-e-gruxe1fico-lc-vs-mt}

A tabela ``Tabela comparativa: LC vs MT'' ilustra as, médias, medianas,
desvios-padrão, notas mínimas e notas máximas das provas de linguagens e
matemática, considerando apenas os alunos que estiveram presentes em
ambas delas.

O gráfico ``comparação das distribuições de notas:LC vs MT'' exibe os
gráficos de distribuição de notas de linguagens e matemática
sobrepostos, oferecendo uma comparação visual direta e clara da
distribuição de notas das duas matérias.

Embora as médias de linguagens e matemática tenham sido muito próximas,
526,7 e 527,2 respectivamente, a distribuição de notas da prova de
matemática é totalmente diferente da de linguagens (que se aproxima de
uma normal perfeita) tendo suas notas muito mais distribuidas no
gráfico, evidenciando um desvio padrão muito maior.

\pandocbounded{\includegraphics[keepaspectratio]{relatorio_files/figure-latex/Gráfico de Densidade LC vs MT-1.pdf}}

O Gráfico de Densidade oferece uma comparação visual direta entre as
distribuições das notas de Linguagens (vermelho) e Matemática (azul).
Este gráfico foi gerado utilizando apenas os alunos que compareceram e
obtiveram nota maior que zero em ambas as provas. Percebe-se que a curva
de Matemática é mais achatada e espalhada (refletindo o maior desvio
padrão), indicando que há uma variabilidade muito maior nas notas. Em
contraste, as notas de Linguagens são mais ``pontudas'' e concentradas
em torno de sua média.

\pandocbounded{\includegraphics[keepaspectratio]{relatorio_files/figure-latex/boxplot-nota-regiao-1.pdf}}

O Boxplot complementa a tabela estatística anterior, visualizando a
distribuição das notas de Linguagens e Códigos por região. Ele foi
criado agrupando os alunos por região e ordenando o eixo Y pela mediana
das notas, da maior para a menor. Este gráfico permite uma visualização
clara não apenas da mediana (a linha central na caixa), mas também da
dispersão do ``meio'' dos alunos (o tamanho da caixa, ou Intervalo
Interquartil) e dos outliers (pontos). Observa-se que as regiões Sudeste
e Sul apresentam as medianas mais elevadas, enquanto as regiões Nordeste
e Norte mostram um desempenho mediano inferior e uma dispersão de notas
(tamanho da caixa) ligeiramente maior, indicando maior variação no
desempenho dos alunos dessas regiões.

\begin{table}
\centering
\begin{tabular}{lrrrrrrr}
\toprule
Região & Média & Mediana & Var & D. Padrão & Min & Máx & Nº de Alunos\\
\midrule
S & 542.13 & 548.0 & 3935.23 & 62.73 & 298.8 & 795.8 & 141162\\
CO & 527.52 & 533.5 & 4542.66 & 67.40 & 298.8 & 777.4 & 24755\\
NE & 510.60 & 516.3 & 4889.35 & 69.92 & 298.8 & 795.8 & 116314\\
N & 499.47 & 505.4 & 4805.12 & 69.32 & 298.8 & 732.3 & 34406\\
\bottomrule
\end{tabular}
\end{table}

\section{Tabela e boxplot de notas LC por
região}\label{tabela-e-boxplot-de-notas-lc-por-regiuxe3o}

A tabela ``Estatísticas das notas de LC por Região'' exibe as médias,
medianas, variências, desvios-padrão, notas mínimas, notas máximas e
número aproximado de alunos de cada região do Brasil.

O ``Boxplot das notas de linguagens por Região'', ilustra a performance
dos alunos de cada uma das regiões do Brasil exibindo a distribuição das
notas. As ``caixas'' representam os 50\% centrais dos alunos, com os
pontos sendo os ``outliers'', ou seja, os fora da média, tanto para
baixo quanto para cima e a linha dentro da caixa representa a mediana.
Com base no boxplot, a região Sudoeste obteve o melhor resultado, com
sua média de 542,74 sendo a mais alta seguida de perto pela região Sul e
sua média de 540.36, o boxplot também deixa evidente a desigualdade do
país, com as regiões Norte e nordeste ficando significativamente atras
das demais.

\end{document}
