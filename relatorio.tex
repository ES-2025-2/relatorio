% Options for packages loaded elsewhere
\PassOptionsToPackage{unicode}{hyperref}
\PassOptionsToPackage{hyphens}{url}
\documentclass[
  twocolumn]{article}
\usepackage{xcolor}
\usepackage[margin=1in]{geometry}
\usepackage{amsmath,amssymb}
\setcounter{secnumdepth}{-\maxdimen} % remove section numbering
\usepackage{iftex}
\ifPDFTeX
  \usepackage[T1]{fontenc}
  \usepackage[utf8]{inputenc}
  \usepackage{textcomp} % provide euro and other symbols
\else % if luatex or xetex
  \usepackage{unicode-math} % this also loads fontspec
  \defaultfontfeatures{Scale=MatchLowercase}
  \defaultfontfeatures[\rmfamily]{Ligatures=TeX,Scale=1}
\fi
\usepackage{lmodern}
\ifPDFTeX\else
  % xetex/luatex font selection
\fi
% Use upquote if available, for straight quotes in verbatim environments
\IfFileExists{upquote.sty}{\usepackage{upquote}}{}
\IfFileExists{microtype.sty}{% use microtype if available
  \usepackage[]{microtype}
  \UseMicrotypeSet[protrusion]{basicmath} % disable protrusion for tt fonts
}{}
\makeatletter
\@ifundefined{KOMAClassName}{% if non-KOMA class
  \IfFileExists{parskip.sty}{%
    \usepackage{parskip}
  }{% else
    \setlength{\parindent}{0pt}
    \setlength{\parskip}{6pt plus 2pt minus 1pt}}
}{% if KOMA class
  \KOMAoptions{parskip=half}}
\makeatother
\usepackage{graphicx}
\makeatletter
\newsavebox\pandoc@box
\newcommand*\pandocbounded[1]{% scales image to fit in text height/width
  \sbox\pandoc@box{#1}%
  \Gscale@div\@tempa{\textheight}{\dimexpr\ht\pandoc@box+\dp\pandoc@box\relax}%
  \Gscale@div\@tempb{\linewidth}{\wd\pandoc@box}%
  \ifdim\@tempb\p@<\@tempa\p@\let\@tempa\@tempb\fi% select the smaller of both
  \ifdim\@tempa\p@<\p@\scalebox{\@tempa}{\usebox\pandoc@box}%
  \else\usebox{\pandoc@box}%
  \fi%
}
% Set default figure placement to htbp
\def\fps@figure{htbp}
\makeatother
\setlength{\emergencystretch}{3em} % prevent overfull lines
\providecommand{\tightlist}{%
  \setlength{\itemsep}{0pt}\setlength{\parskip}{0pt}}
\usepackage{multicol}
\setlength{\columnsep}{0.7cm}
\usepackage{booktabs}
\usepackage{longtable}
\usepackage{array}
\usepackage{multirow}
\usepackage{wrapfig}
\usepackage{float}
\usepackage{colortbl}
\usepackage{pdflscape}
\usepackage{tabu}
\usepackage{threeparttable}
\usepackage{threeparttablex}
\usepackage[normalem]{ulem}
\usepackage{makecell}
\usepackage{xcolor}
\usepackage{bookmark}
\IfFileExists{xurl.sty}{\usepackage{xurl}}{} % add URL line breaks if available
\urlstyle{same}
\hypersetup{
  pdftitle={Relatório},
  pdfauthor={Fabio Firanzi, Heitor Dias, Julia Fideles, Matheus Soares, Tiago Braga},
  hidelinks,
  pdfcreator={LaTeX via pandoc}}

\title{Relatório}
\author{Fabio Firanzi, Heitor Dias, Julia Fideles, Matheus Soares, Tiago
Braga}
\date{2025-11-11}

\begin{document}
\maketitle

\textbf{Tabela 1:} Frequência - Presença na Prova de Matemática

\begin{tabular}{llr}
\toprule
Presença na prova & Percentual & Frequência\\
\midrule
Ausente & 30.56\% & 132415\\
Presente & 69.38\% & 300640\\
Eliminado & 0.06\% & 239\\
\bottomrule
\end{tabular}

\textbf{Gráfico 1:} Presença na Prova

\includegraphics[width=\linewidth]{relatorio-figs/grafico-presenca-1}

\subsection{1. Análise da Presença na Prova de
Matemática}\label{anuxe1lise-da-presenuxe7a-na-prova-de-matemuxe1tica}

A Tabela 1 mostra que 30,56\% dos inscritos não compareceram, enquanto
69,38\% estiveram presentes e apenas 0,06\% foram eliminados. Essa
discrepância revela dois pontos centrais:

\subsubsection{1.1 Comparação entre presença e
ausência}\label{comparauxe7uxe3o-entre-presenuxe7a-e-ausuxeancia}

A quantidade de presentes é mais do que o dobro da de ausentes,
indicando que, embora a taxa de ausência seja significativa, a maior
parte dos estudantes permanece engajada e comparece à prova.

\subsubsection{1.2 Ausências ainda são um desafio
estrutural}\label{ausuxeancias-ainda-suxe3o-um-desafio-estrutural}

Com mais de 132 mil ausentes, observa-se uma dificuldade que pode estar
associada a:

\begin{itemize}
\item
  Logística de deslocamento
\item
  Realização da prova em dia único
\item
  Desmotivação ou insegurança com o exame
\end{itemize}

O gráfico 1 reforça visualmente essa diferença, tornando evidente o
contraste entre o volume de presentes e ausentes.

\begin{center}\rule{0.5\linewidth}{0.5pt}\end{center}

\textbf{Tabela 2:} Provas Padrão vs.~Acessibilidade

\begin{tabular}{llr}
\toprule
Categoria & Percentual & Frequência\\
\midrule
Prova com acessibilidade & 0.22\% & 676\\
Prova padrão & 99.78\% & 299964\\
\bottomrule
\end{tabular}

\textbf{Gráfico 2:} Proporção de Provas com Acessibilidade

\includegraphics[width=\linewidth]{relatorio-figs/grafico-acessibilidade-1}

\subsection{2. Provas Padrão vs.~Provas com
Acessibilidade}\label{provas-padruxe3o-vs.-provas-com-acessibilidade}

A Tabela 2 mostra que 99,78\% dos candidatos fizeram prova padrão,
enquanto apenas 0,22\% demandaram acessibilidade (cerca de 676
estudantes.)

\subsubsection{2.1 Proporção extremamente baixa de necessidades
especiais}\label{proporuxe7uxe3o-extremamente-baixa-de-necessidades-especiais}

Esse valor sugere duas possíveis interpretações:

\begin{itemize}
\item
  O número real de estudantes com necessidades específicas pode ser
  baixo.
\item
  Ou há subregistro de solicitação de acessibilidade, o que é comum em
  exames de grande porte, seja por desconhecimento ou receio de
  burocracias.
\end{itemize}

\subsubsection{2.2 Distribuição interna entre tipos de
acessibilidade}\label{distribuiuxe7uxe3o-interna-entre-tipos-de-acessibilidade}

O gráfico da página 1 revela que, dentro das provas acessíveis:

\begin{itemize}
\item
  Ampliada é a modalidade mais comum (quase metade dos casos).
\item
  Adaptada Ledor, Superampliada e Videoprova Libras aparecem em segundo
  plano.
\item
  Libras -- Ampliada não ocorreu, indicando pouca demanda ou ausência de
  candidatos elegíveis.
\end{itemize}

Essa distribuição mostra que deficiência visual é o principal motivo
para solicitação de acessibilidade, seguido de necessidades pedagógicas
específicas.

\begin{center}\rule{0.5\linewidth}{0.5pt}\end{center}

\textbf{Tabela 3:} Frequência Intervalar da Nota de Matemática

\begin{tabular}{ll}
\toprule
Intervalo de nota & Percentual\\
\midrule
0 – 100 & 0.03\%\\
100 – 200 & 0\%\\
200 – 300 & 0\%\\
300 – 400 & 10.1\%\\
400 – 500 & 40.17\%\\
\addlinespace
500 – 600 & 22.48\%\\
600 – 700 & 18.46\%\\
700 – 800 & 7.06\%\\
800 – 900 & 1.52\%\\
900 – 1000 & 0.18\%\\
\bottomrule
\end{tabular}

\subsection{3. Distribuição das Notas de
Matemática}\label{distribuiuxe7uxe3o-das-notas-de-matemuxe1tica}

A Tabela 3 divide as notas em intervalos de 100 pontos, permitindo
observar o comportamento geral do desempenho dos estudantes.

\subsubsection{3.1 Concentração forte em torno do meio da
distribuição}\label{concentrauxe7uxe3o-forte-em-torno-do-meio-da-distribuiuxe7uxe3o}

Os intervalos com maiores proporções são:

\begin{itemize}
\item
  400--500: 40,17\%
\item
  500--600: 22,48\%
\end{itemize}

Somando esses dois grupos, mais de 62\% dos participantes estão entre
400 e 600 pontos, indicando que:

\begin{itemize}
\item
  A maior parte dos estudantes teve desempenho mediano, nem muito baixo,
  nem muito alto.
\item
  A dificuldade da prova parece adequadamente calibrada para centralizar
  alunos em torno da média.
\end{itemize}

\subsection{3.2 Baixa incidência de notas
extremas}\label{baixa-inciduxeancia-de-notas-extremas}

Notas muito baixas (\textless300) e muito altas (\textgreater800)
representam percentual muito pequeno:

\begin{itemize}
\item
  \textless100 pontos: 0,03\%
\item
  800--900: 1,52\%
\item
  900--1000: 0,18\%
\end{itemize}

Esses valores sugerem que a prova discrimina bem alunos com domínio
intermediário, mas raramente produz desempenhos extremos.

\begin{center}\rule{0.5\linewidth}{0.5pt}\end{center}

\textbf{Gráfico 3:} Histograma das Notas de Matemática com Curva Normal

\includegraphics[width=\linewidth]{relatorio-figs/hist-nota-mt-normal-1}

\subsection{4. Ajuste à Curva Normal da
Distribuição}\label{ajuste-uxe0-curva-normal-da-distribuiuxe7uxe3o}

O histograma com a curva normal ajustada mostra que:

\subsubsection{4.1 A distribuição é aproximadamente
normal}\label{a-distribuiuxe7uxe3o-uxe9-aproximadamente-normal}

A forma do gráfico indica:

\begin{itemize}
\item
  Picos de densidade próximos ao centro
\item
  Decaimento simétrico nos extremos
\item
  Ausência de longas caudas que caracterizariam assimetria acentuada
\end{itemize}

Isso sugere que:

\begin{itemize}
\item
  O exame conseguiu distribuir os alunos segundo um padrão próximo ao
  esperado estatisticamente.
\item
  Há equilíbrio entre questões fáceis, intermediárias e difíceis.
\end{itemize}

4.2 Diferença entre curva teórica e real

Pequenas diferenças entre a curva real e a curva normal ajustada indicam
que:

\begin{itemize}
\tightlist
\item
  Alguns níveis de nota são mais frequentes que o previsto pela normal
  (natural em testes educacionais).
\end{itemize}

\end{document}
