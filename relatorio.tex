% Options for packages loaded elsewhere
\PassOptionsToPackage{unicode}{hyperref}
\PassOptionsToPackage{hyphens}{url}
\documentclass[
  twocolumn]{article}
\usepackage{xcolor}
\usepackage[margin=1in]{geometry}
\usepackage{amsmath,amssymb}
\setcounter{secnumdepth}{-\maxdimen} % remove section numbering
\usepackage{iftex}
\ifPDFTeX
  \usepackage[T1]{fontenc}
  \usepackage[utf8]{inputenc}
  \usepackage{textcomp} % provide euro and other symbols
\else % if luatex or xetex
  \usepackage{unicode-math} % this also loads fontspec
  \defaultfontfeatures{Scale=MatchLowercase}
  \defaultfontfeatures[\rmfamily]{Ligatures=TeX,Scale=1}
\fi
\usepackage{lmodern}
\ifPDFTeX\else
  % xetex/luatex font selection
\fi
% Use upquote if available, for straight quotes in verbatim environments
\IfFileExists{upquote.sty}{\usepackage{upquote}}{}
\IfFileExists{microtype.sty}{% use microtype if available
  \usepackage[]{microtype}
  \UseMicrotypeSet[protrusion]{basicmath} % disable protrusion for tt fonts
}{}
\makeatletter
\@ifundefined{KOMAClassName}{% if non-KOMA class
  \IfFileExists{parskip.sty}{%
    \usepackage{parskip}
  }{% else
    \setlength{\parindent}{0pt}
    \setlength{\parskip}{6pt plus 2pt minus 1pt}}
}{% if KOMA class
  \KOMAoptions{parskip=half}}
\makeatother
\usepackage{graphicx}
\makeatletter
\newsavebox\pandoc@box
\newcommand*\pandocbounded[1]{% scales image to fit in text height/width
  \sbox\pandoc@box{#1}%
  \Gscale@div\@tempa{\textheight}{\dimexpr\ht\pandoc@box+\dp\pandoc@box\relax}%
  \Gscale@div\@tempb{\linewidth}{\wd\pandoc@box}%
  \ifdim\@tempb\p@<\@tempa\p@\let\@tempa\@tempb\fi% select the smaller of both
  \ifdim\@tempa\p@<\p@\scalebox{\@tempa}{\usebox\pandoc@box}%
  \else\usebox{\pandoc@box}%
  \fi%
}
% Set default figure placement to htbp
\def\fps@figure{htbp}
\makeatother
\setlength{\emergencystretch}{3em} % prevent overfull lines
\providecommand{\tightlist}{%
  \setlength{\itemsep}{0pt}\setlength{\parskip}{0pt}}
\usepackage{multicol}
\setlength{\columnsep}{0.7cm}
\usepackage[T1]{fontenc} \usepackage[utf8]{inputenc} \usepackage[portuguese]{babel} \usepackage{booktabs} \usepackage{float} \usepackage{graphicx}
\usepackage{booktabs}
\usepackage{longtable}
\usepackage{array}
\usepackage{multirow}
\usepackage{wrapfig}
\usepackage{float}
\usepackage{colortbl}
\usepackage{pdflscape}
\usepackage{tabu}
\usepackage{threeparttable}
\usepackage{threeparttablex}
\usepackage[normalem]{ulem}
\usepackage{makecell}
\usepackage{xcolor}
\usepackage{bookmark}
\IfFileExists{xurl.sty}{\usepackage{xurl}}{} % add URL line breaks if available
\urlstyle{same}
\hypersetup{
  pdftitle={Relatorio},
  pdfauthor={Fabio Firanzi, Heitor Dias, Julia Fideles, Matheus Soares, Tiago Braga},
  hidelinks,
  pdfcreator={LaTeX via pandoc}}

\title{Relatorio}
\author{Fabio Firanzi, Heitor Dias, Julia Fideles, Matheus Soares, Tiago
Braga}
\date{2025-11-11}

\begin{document}
\maketitle

\section{Análise das variáveis de Ciências da
Natureza}\label{anuxe1lise-das-variuxe1veis-de-ciuxeancias-da-natureza}

\noindent

\begin{minipage}[t]{0.48\textwidth}
\centering
\textbf{Tabela 1:} Frequência - Presença na Prova de Ciências da Natureza
\par\vspace{1ex}
{\footnotesize

\begin{tabular}{llr}
\toprule
Presença na prova & Percentual & Frequência\\
\midrule
Ausente & 30.56\% & 132415\\
Presente & 69.38\% & 300640\\
Eliminado & 0.06\% & 239\\
\bottomrule
\end{tabular}
}
\\
A partir da Tabela 1, observa-se a distribuição dos participantes em relação à presença na prova de Ciências da Natureza, destacando-se a proporção de estudantes que compareceram, se ausentaram ou foram eliminados. Essa informação é importante para entender o engajamento dos inscritos na aplicação da prova.
\end{minipage}
\hfill
\begin{minipage}[t]{0.48\textwidth}
\centering
\textbf{Gráfico 1:} Presença na Prova de Ciências da Natureza
\par\vspace{1ex}


\includegraphics[width=\linewidth]{relatorio-figs/grafico-presenca-cn-1} 
O Gráfico 1 apresenta visualmente a comparação entre estudantes presentes, ausentes e eliminados na prova de Ciências da Natureza, facilitando a interpretação da participação na avaliação.
\end{minipage}

\vfill

\noindent

\begin{minipage}[t]{0.48\textwidth}
\centering
\textbf{Tabela 2:} Frequência Intervalar da Nota de Ciências da Natureza
\par\vspace{1ex}
{\footnotesize

\begin{tabular}{ll}
\toprule
Intervalo de nota & Percentual\\
\midrule
0 – 100 & 0.02\%\\
100 – 200 & 0\%\\
200 – 300 & 0\%\\
300 – 400 & 12.32\%\\
400 – 500 & 42.64\%\\
\addlinespace
500 – 600 & 35.04\%\\
600 – 700 & 9.28\%\\
700 – 800 & 0.69\%\\
800 – 900 & 0.01\%\\
900 – 1000 & 0\%\\
\bottomrule
\end{tabular}
}
\\
A Tabela 2 apresenta a distribuição das notas de Ciências da Natureza em classes de 100 pontos, permitindo identificar em quais faixas se concentram os maiores e os menores desempenhos dos estudantes.
\end{minipage}
\hfill
\begin{minipage}[t]{0.48\textwidth}
\centering
\textbf{Gráfico 2:} Histograma das Notas de Ciências da Natureza com Curva Normal
\par\vspace{1ex}

\includegraphics[width=\linewidth]{relatorio-figs/hist-nota-cn-normal-1} 
\\
O Gráfico 2 compara a distribuição relativa das notas de Ciências da Natureza com uma curva normal ajustada pela média e pelo desvio padrão da amostra, permitindo avaliar se o desempenho se aproxima de um padrão aproximadamente normal ou se há assimetrias importantes.
\end{minipage}

\end{document}
