% Options for packages loaded elsewhere
\PassOptionsToPackage{unicode}{hyperref}
\PassOptionsToPackage{hyphens}{url}
\documentclass[
  twocolumn]{article}
\usepackage{xcolor}
\usepackage[margin=1in]{geometry}
\usepackage{amsmath,amssymb}
\setcounter{secnumdepth}{-\maxdimen} % remove section numbering
\usepackage{iftex}
\ifPDFTeX
  \usepackage[T1]{fontenc}
  \usepackage[utf8]{inputenc}
  \usepackage{textcomp} % provide euro and other symbols
\else % if luatex or xetex
  \usepackage{unicode-math} % this also loads fontspec
  \defaultfontfeatures{Scale=MatchLowercase}
  \defaultfontfeatures[\rmfamily]{Ligatures=TeX,Scale=1}
\fi
\usepackage{lmodern}
\ifPDFTeX\else
  % xetex/luatex font selection
\fi
% Use upquote if available, for straight quotes in verbatim environments
\IfFileExists{upquote.sty}{\usepackage{upquote}}{}
\IfFileExists{microtype.sty}{% use microtype if available
  \usepackage[]{microtype}
  \UseMicrotypeSet[protrusion]{basicmath} % disable protrusion for tt fonts
}{}
\makeatletter
\@ifundefined{KOMAClassName}{% if non-KOMA class
  \IfFileExists{parskip.sty}{%
    \usepackage{parskip}
  }{% else
    \setlength{\parindent}{0pt}
    \setlength{\parskip}{6pt plus 2pt minus 1pt}}
}{% if KOMA class
  \KOMAoptions{parskip=half}}
\makeatother
\usepackage{graphicx}
\makeatletter
\newsavebox\pandoc@box
\newcommand*\pandocbounded[1]{% scales image to fit in text height/width
  \sbox\pandoc@box{#1}%
  \Gscale@div\@tempa{\textheight}{\dimexpr\ht\pandoc@box+\dp\pandoc@box\relax}%
  \Gscale@div\@tempb{\linewidth}{\wd\pandoc@box}%
  \ifdim\@tempb\p@<\@tempa\p@\let\@tempa\@tempb\fi% select the smaller of both
  \ifdim\@tempa\p@<\p@\scalebox{\@tempa}{\usebox\pandoc@box}%
  \else\usebox{\pandoc@box}%
  \fi%
}
% Set default figure placement to htbp
\def\fps@figure{htbp}
\makeatother
\setlength{\emergencystretch}{3em} % prevent overfull lines
\providecommand{\tightlist}{%
  \setlength{\itemsep}{0pt}\setlength{\parskip}{0pt}}
\usepackage{multicol}
\setlength{\columnsep}{0.7cm}
\usepackage[T1]{fontenc} \usepackage[utf8]{inputenc} \usepackage[portuguese]{babel} \usepackage{booktabs} \usepackage{float} \usepackage{graphicx}
\usepackage{booktabs}
\usepackage{longtable}
\usepackage{array}
\usepackage{multirow}
\usepackage{wrapfig}
\usepackage{float}
\usepackage{colortbl}
\usepackage{pdflscape}
\usepackage{tabu}
\usepackage{threeparttable}
\usepackage{threeparttablex}
\usepackage[normalem]{ulem}
\usepackage{makecell}
\usepackage{xcolor}
\usepackage{bookmark}
\IfFileExists{xurl.sty}{\usepackage{xurl}}{} % add URL line breaks if available
\urlstyle{same}
\hypersetup{
  pdftitle={Relatorio},
  pdfauthor={Fabio Firanzi, Heitor Dias, Julia Fideles, Matheus Soares, Tiago Braga},
  hidelinks,
  pdfcreator={LaTeX via pandoc}}

\title{Relatorio}
\author{Fabio Firanzi, Heitor Dias, Julia Fideles, Matheus Soares, Tiago
Braga}
\date{2025-11-11}

\begin{document}
\maketitle

\begin{tabular}{llr}
\toprule
Língua Estrangeira & Percentual & Frequência\\
\midrule
Inglês & 54.93\% & 237997\\
Espanhol & 45.07\% & 195297\\
\bottomrule
\end{tabular}

Observa-se, a partir da tabela, que aproximadamente 55\% dos
participantes escolheu a opção ``inglês'' para a prova de língua
estrangeira e 45\% dos participantes escolheu ``espanhol'', o que
evidencia que a maior parte dos estudantes optou pela língua inglesa.

\includegraphics[width=\linewidth]{relatorio-figs/grafico-linguaEstrangeira-1}
O gráfico apresenta, de forma imagética, o contraste entre a quantidade
do todo que optou pela língua inglesa e a quantidade que optou pela
língua espanhola.

\textbf{Tabela 2:} Frequência - Motivo do Zero na Redação

\begin{tabular}{ll}
\toprule
Motivo & Percentual\\
\midrule
Anulada & 1.58\%\\
Cópia Texto Motivador & 11.01\%\\
Em Branco & 54.2\%\\
Fuga ao Tema & 19.97\%\\
Não atendimento ao tipo textual & 1.99\%\\
\addlinespace
Texto insuficiente & 8.54\%\\
Parte desconectada & 2.7\%\\
\bottomrule
\end{tabular}

Observando a tabela, percebe-se que mais da metade (54,2\%) das redações
zeradas teve como motivo a redação em branco, sendo o principal motivo
para isso. Os demais motivos, em ordem após esse, são fuga ao tema
(19,97\%), cópia do texto motivador (11,01\%), texto insuficiente
(8,54\%), parte desconectada (2,7\%), não atendimento ao tipo textual
(1,99\%) e anulação (1,58\%). Com isso, é possível constatar que, a
maior parte dos participantes que teve a redação zerada não escreveu ou
escreveu insuficientemente (54,2\% em branco e 8,54\% texto
insuficiente), não entendeu o tema ou não foi capaz de relacionar o tema
a conhecimentos prévios e copiou os textos motivadores.

\textbf{Gráfico 2:} Motivos para Redação Zerada

\includegraphics[width=\linewidth]{relatorio-figs/grafico-motivoZero-1}
O gráfico, apresenta de forma imagética, a relação entre os motivos das
redações zeradas e a frequência relativa deles. É possível observar que
mais da metade das redações foi zerada por um motivo (Em Branco) e a
outra metade foi por motivos diversos, o que evidencia que um motivo se
sobressai aos demais.

\textbf{Tabela 3:} Frequência - Notas nas Competências

\begin{tabular}{llllll}
\toprule
Nota & C1 & C2 & C3 & C4 & C5\\
\midrule
0 & 0.01\% & 0\% & 0.02\% & 0.02\% & 6.67\%\\
20 & 0.01\% & 0\% & 0.03\% & 0.04\% & 2.83\%\\
40 & 0.21\% & 0.8\% & 1.81\% & 0.35\% & 4.49\%\\
60 & 0.96\% & 0.54\% & 2.74\% & 1.49\% & 4.57\%\\
80 & 7.78\% & 2.03\% & 10.38\% & 7.34\% & 10.98\%\\
\addlinespace
100 & 13.92\% & 3.55\% & 13.08\% & 10.85\% & 10.31\%\\
120 & 37.04\% & 27.06\% & 34.67\% & 32.03\% & 15.22\%\\
140 & 17.12\% & 11.24\% & 13.59\% & 12.97\% & 9.66\%\\
160 & 21.53\% & 18.89\% & 14.87\% & 14.99\% & 11.82\%\\
180 & 1.35\% & 17.39\% & 6.52\% & 10.55\% & 9.27\%\\
\addlinespace
200 & 0.07\% & 18.51\% & 2.28\% & 9.37\% & 14.19\%\\
\bottomrule
\end{tabular}

A tabela apresenta a relação entre as notas e a frequência relativa da
aparição da mesma em cada uma das 5 competências da redação, em redações
que não foram zeradas. Ao comparar as frequências das notas em todas as
competências, as principais constatações são: ``A competência com maior
taxa de 0 é a competência 5 (6,67\%)'', ``A competência 1 possui a maior
parte das notas concentradas entre 100 e 160 (com porcentagens mais
significativas, entre 13,92\% e 37,04\%)'', ``A competência 1 é a única
cuja taxa de notas 200 é significativamente menor'', ``A competência 5
possui maior distribuição de notas que as demais (todas as porcentagens
estão acima de 2\%)'' e ``A competência 2 é a competência com maior taxa
de notas entre 120 e 200 (93,09\% das notas está nessa faixa)''.

\textbf{Gráfico 3:} Distribuição das Notas por Competência

\includegraphics[width=\linewidth]{relatorio-figs/boxplot-notaCompetencias-1}
O boxplot representa, graficamente, as informações presentes na tabela.
Os pontos indicam valores que estão fora da concentração de resultados,
mas que tiveram aparição (outliers). As principais constatações são: A
competência 1 é a que possui menor dispersão, ou seja, a maioria dos
resultados está concentrada em uma faixa pequena (100 a 160) e possui
outliers em todas as notas fora dessa faixa. A competência 2 possui
concentração de notas mais altas que as demais, o que é evidenciado por
abranger notas mais altas e não possui outliers, ou seja, todos os
resultados estão na faixa estipulada. A competência 5 possui maior
dispersão, ou seja, abrange todas as notas possíveis e está mais
distribuida.

\textbf{Tabela 4:} Frequência - Notas da Redação

\begin{tabular}{ll}
\toprule
Intervalo de Notas & Percentual\\
\midrule
x <= 100 & 0.02\%\\
100 < x <= 200 & 0.1\%\\
200 < x <= 300 & 1.05\%\\
300 < x <= 400 & 5.37\%\\
400 < x <= 500 & 12.17\%\\
\addlinespace
500 < x <= 600 & 22.43\%\\
600 < x <= 700 & 20.39\%\\
700 < x <= 800 & 16.34\%\\
800 < x <= 900 & 15.52\%\\
900 < x & 6.61\%\\
\bottomrule
\end{tabular}

Média das notas: 659.33

Desvio Padrão: 163.34

A tabela indica a frequência relativa, em porcentagem, das faixas de
notas da redação. Aproximadamente metade dos resultados está centrado
entre 500 e 700 (22,43\% + 20, 39\%), enquanto os demais intervalos têm
frequência cada vez menor conforme se afasta do centro, tendendo para os
intervalos extremos.

\textbf{Gráfico 4:} Distribuição das Notas da Redação

\includegraphics[width=\linewidth]{relatorio-figs/histograma-notaRedacao-1}

O gráfico relaciona o comportamento da frequência dos intervalos das
notas da redação (presentes na tabela 4), indicado pelas barras, com o
comportamento da normal, linha, para a verificação da semelhança entre
ambos. É possível observar que a maior frequência de aparição dos
valores converge para o centro, enquanto diminui ao se aproximar das
extremidades, o que se assemelha ao comportamento teórico da normal.

\end{document}
