% Options for packages loaded elsewhere
\PassOptionsToPackage{unicode}{hyperref}
\PassOptionsToPackage{hyphens}{url}
\documentclass[
]{article}
\usepackage{xcolor}
\usepackage[margin=1in]{geometry}
\usepackage{amsmath,amssymb}
\setcounter{secnumdepth}{-\maxdimen} % remove section numbering
\usepackage{iftex}
\ifPDFTeX
  \usepackage[T1]{fontenc}
  \usepackage[utf8]{inputenc}
  \usepackage{textcomp} % provide euro and other symbols
\else % if luatex or xetex
  \usepackage{unicode-math} % this also loads fontspec
  \defaultfontfeatures{Scale=MatchLowercase}
  \defaultfontfeatures[\rmfamily]{Ligatures=TeX,Scale=1}
\fi
\usepackage{lmodern}
\ifPDFTeX\else
  % xetex/luatex font selection
\fi
% Use upquote if available, for straight quotes in verbatim environments
\IfFileExists{upquote.sty}{\usepackage{upquote}}{}
\IfFileExists{microtype.sty}{% use microtype if available
  \usepackage[]{microtype}
  \UseMicrotypeSet[protrusion]{basicmath} % disable protrusion for tt fonts
}{}
\makeatletter
\@ifundefined{KOMAClassName}{% if non-KOMA class
  \IfFileExists{parskip.sty}{%
    \usepackage{parskip}
  }{% else
    \setlength{\parindent}{0pt}
    \setlength{\parskip}{6pt plus 2pt minus 1pt}}
}{% if KOMA class
  \KOMAoptions{parskip=half}}
\makeatother
\usepackage{graphicx}
\makeatletter
\newsavebox\pandoc@box
\newcommand*\pandocbounded[1]{% scales image to fit in text height/width
  \sbox\pandoc@box{#1}%
  \Gscale@div\@tempa{\textheight}{\dimexpr\ht\pandoc@box+\dp\pandoc@box\relax}%
  \Gscale@div\@tempb{\linewidth}{\wd\pandoc@box}%
  \ifdim\@tempb\p@<\@tempa\p@\let\@tempa\@tempb\fi% select the smaller of both
  \ifdim\@tempa\p@<\p@\scalebox{\@tempa}{\usebox\pandoc@box}%
  \else\usebox{\pandoc@box}%
  \fi%
}
% Set default figure placement to htbp
\def\fps@figure{htbp}
\makeatother
\setlength{\emergencystretch}{3em} % prevent overfull lines
\providecommand{\tightlist}{%
  \setlength{\itemsep}{0pt}\setlength{\parskip}{0pt}}
\usepackage[T1]{fontenc}
\usepackage[utf8]{inputenc}
\usepackage[portuguese]{babel}
\usepackage{booktabs}
\usepackage{float}
\usepackage{graphicx}
\usepackage{booktabs}
\usepackage{longtable}
\usepackage{array}
\usepackage{multirow}
\usepackage{wrapfig}
\usepackage{float}
\usepackage{colortbl}
\usepackage{pdflscape}
\usepackage{tabu}
\usepackage{threeparttable}
\usepackage{threeparttablex}
\usepackage[normalem]{ulem}
\usepackage{makecell}
\usepackage{xcolor}
\usepackage{bookmark}
\IfFileExists{xurl.sty}{\usepackage{xurl}}{} % add URL line breaks if available
\urlstyle{same}
\hypersetup{
  pdftitle={Relatório},
  pdfauthor={Fabio Firanzi, Heitor Dias, Julia Fideles, Matheus Soares, Tiago Braga},
  hidelinks,
  pdfcreator={LaTeX via pandoc}}

\title{Relatório}
\author{Fabio Firanzi, Heitor Dias, Julia Fideles, Matheus Soares, Tiago
Braga}
\date{2025-11-11}

\begin{document}
\maketitle

\noindent

\begin{minipage}[t]{0.48\textwidth}
\centering
\textbf{Tabela 1:} Frequência - Presença na Prova de Matemática
\par\vspace{1ex}
{\footnotesize

\begin{tabular}{llr}
\toprule
Presença na prova & Percentual & Frequência\\
\midrule
Ausente & 30.56\% & 132415\\
Presente & 69.38\% & 300640\\
Eliminado & 0.06\% & 239\\
\bottomrule
\end{tabular}
}
\\
A partir da Tabela 1, observa-se que aproximadamente 30\% dos estudantes inscritos estavam ausentes no dia da prova de matemática. Cerca de 70\% compareceram e aproximadamente 0,06\% foram eliminados. Esses valores evidenciam que a grande maioria dos participantes realizou a avaliação, apesar de uma parcela expressiva de ausentes.
\end{minipage}
\hfill
\begin{minipage}[t]{0.48\textwidth}
\centering
\textbf{Gráfico 1:} Presença na Prova
\par\vspace{1ex}


\includegraphics[width=\linewidth]{relatorio-figs/grafico-presenca-1} 
O Gráfico 1 demonstra de forma visual a comparação da quantidade de estudantes que compareceram ou não à prova ou foram eliminados.
\end{minipage}

\vfill

\noindent

\begin{minipage}[t]{0.48\textwidth}
\centering
\textbf{Tabela 2:} Provas Padrão vs. Acessibilidade
\par\vspace{1ex}
{\footnotesize

\begin{tabular}{llr}
\toprule
Categoria & Percentual & Frequência\\
\midrule
Prova com acessibilidade & 0.22\% & 676\\
Prova padrão & 99.78\% & 299964\\
\bottomrule
\end{tabular}
\\
A partir da tabela 2, conclui-se que somente aproximadamente 0,22\% dos inscritos na prova solicitaram recursos de acessbilidade para a prova de matemática
}
\end{minipage}
\hfill
\begin{minipage}[t]{0.48\textwidth}
\centering
\textbf{Gráfico 2:} Proporção de Provas com Acessibilidade
\par\vspace{1ex}


\includegraphics[width=\linewidth]{relatorio-figs/grafico-acessibilidade-1} 
\\
O Gráfico 2 mostra a distribuição percentual dos tipos de provas com acessibilidade. A modalidade Ampliada é a mais utilizada, seguida pelas versões Adaptada Ledor, Videoprova – Libras e Superampliada. A categoria Libras – Ampliada não apresentou registros na amostra.
\end{minipage}

\end{document}
